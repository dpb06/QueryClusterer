%File: formatting-instruction.tex
\documentclass[letterpaper]{article}
\usepackage{aaai}
\usepackage{times}
\usepackage{helvet}
\usepackage{courier}
\frenchspacing
\setlength{\pdfpagewidth}{8.5in}
\setlength{\pdfpageheight}{11in}
\pdfinfo{
/Title (Insert Your Title Here)
/Author (Put All Your Authors Here, Separated by Commas)}
\setcounter{secnumdepth}{0}  
 \begin{document}
% The file aaai.sty is the style file for AAAI Press 
% proceedings, working notes, and technical reports.
%
\title{Catagory Clustering for Query Classification\thanks{Thanks to Xiaoxing Gao for your help}}
\author{David Blackford\\
Victoria University of Wellington\\
Kelburn Parade\\
Wellington\\
New Zealand\\
}
\maketitle
\begin{abstract}
\begin{quote}
Short Query classification is an work area that relies on the enrichment of short queries in order to get enough information to successfully classify the query into 
one of many catagories. This task can be done in many ways, and this paper discusses my attempt at a clustering algorithm designed to catagorize a query.
\end{quote}
\end{abstract}

\begin{document}


\section{Algorithm}
The algorithm that I designed is based on the work \cite{distCluster} and attempts to find the most common distribution of catagories for words that are related
to a given query. The query begins unenriched, and is enriched through use of the Google search engine(http://www.google.com). The resulting body of words is taken
as related to the query, and these words are used to find a distibution of the algorithms relative belief that the query should belong to a given catagory/catagories.
Each word for the enriched query measured against each of the given catagories using one of many possible measures. Once we have measured the rough probability distribution of the word
belonging to the different catagories, we have to decide on which of these distributions comes closest to representing the query which the words are enriching. 

In this algorithm we do so by, rather than selecting a distribution, by clustering similar distributions until one distribution becomes a fair representation of the majority of the distributions
we began with. This is completed throught use of the Kullbeck-Leibler algorithm \cite{kullbeckLeibler} which is an algorithm which calculates a measure, the dissimilarity, between two given distributions. If two distributions are similar, then the dissimilarity value returned by the Kullbeck-Leibler algorithm will be extremely low. If two distributions are completely unlike each other, then both the similarity

\section{Implementation}

Written in java etc
The measure that was originally attempted for this algorithm was the relatedness measure found from Wikiminer \cite{wikiminer} however the implementation of the Wikiminer program was poorly documented, and the relatedness measure relied on a webservice that seemed to be shut-down. Another measure that was attempted was use of Normalized Google Distance \cite{NGD}. This is a measure which makes use of the 
Google search engine's approximate pages in order to calculate how often two words appear together relative to how often they appear apart. This measure seems extremely promising, however
Google places limits on the number of queries and frequency of queries. For initial 
\bibliographystyle{aaai}
\bibliography{bibfile1}
\end{document}
